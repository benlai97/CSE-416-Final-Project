%
% Generic Template
% Created on 11/6/2017 by Jonathan Chen
%

\documentclass[12pt]{article}
\usepackage[left=0.8in,right=0.8in,top=0.8in,bottom=0.8in]{geometry}
\usepackage[utf8]{inputenc}
\usepackage{fancyhdr}                  % for header
\usepackage{graphicx}                  % for figures
\usepackage{amsmath}                   % for extented math markup
\usepackage{amssymb}
\usepackage{mathtools}
\usepackage{minted}
\usepackage{enumitem}
\usepackage{csquotes}
\usepackage{caption}
\usepackage{lipsum}
\usepackage[bookmarks=false]{hyperref} % for URL embedding
\usepackage{hyperref}

%%%%%%%%%%%%%%%%%%%%%%%%%%%%%%%%%%%%%%%%%%%%%%%%%%%%%%%%%%%%%%%%%%%%%%%%%%%%%%%%

\newcommand{\course}{CSE416A}
\newcommand{\names}{Benjamin Lai, Cate Jiang, and Jonathan Chen}

% create sheader and footer for every page
\pagestyle{fancy}
\fancyhf{}
\lhead{\textbf{Legislative Complex Networks}}
\chead{}
\rhead{Lai, Jiang, and Chen}
\cfoot{\thepage}

% environment for cases
\newlength\mylen
\settowidth\mylen{\textbf{Case~5.}}
\newlist{enumcases}{enumerate}{1}
\setlist[enumcases,1]{label=\textbf{Case~\arabic*.},leftmargin=0pt,align=right,%
    labelwidth=\dimexpr-\mylen-\labelsep\relax,}

% images
\captionsetup[figure]{
  font={footnotesize,singlespacing},
  labelfont=bf,
  singlelinecheck=off,
  justification=raggedright,
  labelsep=period
}

\newcommand{\image}[5][0.6]{
  % 1 - path to asset
  % 2 - label
  % 3 - title
  % 4 - caption
  \begin{figure}[H]
    \centering
    \includegraphics[width=1.0\textwidth,height=#1\textwidth,keepaspectratio]{#2}
    \caption[(details)]{\textbf{#4}. #5}
    \label{fig:#3}
  \end{figure}}


% minted
\usemintedstyle{pastie}
\setminted[python]{frame=lines,linenos,autogobble,framesep=1.4em,breaklines,%
                   escapeinside=||,mathescape,python3}

%%%%%%%%%%%%%%%%%%%%%%%%%%%%%%%%%%%%%%%%%%%%%%%%%%%%%%%%%%%%%%%%%%%%%%%%%%%%%%%%
% the document begins here!

\title{Comparative Analysis of Legislative Complex Networks of Select North American and European Countries}
\author{\names \\ \small\url{https://github.com/jowch/CSE-416-Final-Project}}
  
\date{December 11, 2018}
    
\begin{document}
\maketitle
% \vspace{0.1em}
\thispagestyle{empty}

\begin{abstract}
    Given the complex nature of today's political climate, we were interested in analyzing and quantifying various legislative bodies in select first-world countries. To do so, we performed complex network analysis on seven North Amerian and European countries (Austria, Denmark, France, Sweden, Switerland, UK, and US) to understand how a wide variety political parties interacted amongst each other with respect to cosponsorship. We found and plotted summary statistics and centrality measures to make comparisons possible.
\end{abstract}

\section{Introduction} 

In the United States, politics has increasingly become a point of conversation due to change of presidency and new bills being signed into laws. Cosponsorship provides a rich network between legislators that can be used to explain the political effects and efficiency of legislative processes. We propose to analyze the networks of different legislative bodies in select North American and European countries to determine how legislators interact amongst each other to cosponsor bills. We aim to answer questions such as which countries' legislators are most connected, how legislative parties and other attributes play into cosponsorship data, and what that means in terms of government performance and productivity. To perform our research, we will use cosponsorship data from different countries followed by analysis and comparison of statistics and centrality measures.

According a study conducted by James Folwer from the Department of Political Science at the University of California, Davis, there are several trends that have been observed among legislators. In the United States, biennial elections cause members of the U.S. House and Senate to change every two years, but the trends remain stable. Firstly, party affiliation and similar values and ideals often lead to increased possibility of cosponsorship. Second, average number of cosponsors per legislator reflects the degree that the individual is integrated into their political network \cite{fowler1}. Because these studies were performed solely within U.S. Congression, we propose to analyze legislators from different first-world countries to determine if these trends would be observed in other political environments as well.
 
\section{Materials and Methods}

\subsection{Data Collection and Pre-Processing}
We first obtained cosponsorship data for the countries: Austria, Denmark, France, Sweden, Switzerland, United Kingdom, and the United States. This was taken taken from the following resources: \underline{Bill cosponshorship} \cite{briatte}, \underline{Cosponsorship Network} \cite{fowler}, \underline{UK Parliament} \cite{UK}, and \underline{US Congress} \cite{US}. The raw data was parsed and saved as \texttt{graphml} files. Because the data from many databases were divided based on years, we also had to load and combine various \texttt{graphml} files to create a single network for each country we were analyzing. The data is stored in \texttt{CSE-416-Final-Project/data/} and the pre-processing was performed in the \texttt{Analysis} file in the root directory within our repository.

%TODO: scraping

\subsection{Data Analysis}

Our analysis was performed using a library called \href{https://graph-tool.skewed.de}{graph-tool}, which provides Python bindings to powerful abstractions and parallel algorithms for processing graphs. In addition, we used \texttt{numpy}, \texttt{scipy}, \texttt{pandas}, \texttt{seaborn}, and \texttt{matplotlib} to perform the comparative analysis.

We wrote a summary function that returns a summary of graph statistics with a given graph. These statistics include: number of edges, number of vertices, average clustering coefficient, average degree, average excess degree, average path length, size of largest connected component, and number of connected components. We then wrote a script that produces a summary output in a csv file for every country for every year. 

An important attribute of our generated networks that we needed to measure was assortativity. Each graph had various vertex properties to perform assortativity measures on, but we decided to use `party' to determine whether governments cooporate on a partison or bipartison basis. We mapped each `party' to a unique ID, and joined this party enumeration property map with the original vertex properties. We then produced graphs for every country across each year, with nodes colored based off their respective enumerated `party' property map.

Betweenness, closeness, and eigenvector centrality measures were also performed on our data to identify influential legislators. The graphed vertices and edges were scaled and colored with respect to its corresponding centrality measure. Vertices and edges that are yellow and larger in size reflect greater importance, whilst those that are blue and smaller in size reflect lesser importance. For the generated betweenness graphs, the top 3 percent of legislators with the highest betweenness centralities were colored yellow and enlarged.


\section{Results}
\subsection{Country-level Results}
\begin{itemize}
  \item Every country we look at has some sort of temporal data except for Denmark and the UK. Explain why we don't have the data because of how the database worked. But we still performed the same summary statistics and produced graphs.
  \item Austria - the number of edges and vertices do not fluctuate between the years for most countries. Number of vertices around 63. Average degree has been relatively stable, which is expected given the lack of change in edges and vertices. Excess degree decreases from 1995 to 1999. The excess degree dropped to around 2, which is similar to the average degree for Austria. This happens to coincide with Austria joining the European Union in 1995. It is unclear whether there is a correlation between joining the EU and the sharp decrease in average excess degree. However, it can be noted that the drop in excess degrees leads to a similar value as the average degree, which means that the degree of individuals are roughly the same as the degree of their neighbors on average. This also means there is less or almost no hub behavior.
  % To-do: clustering coefficient is 0 for years 94-95, 06-08, 13-18 but we don't know why -- show plots
  \item France has had a steady increase of vertices. It started with a little less than 300 verticies in 1986-1988, and now it is around 500. Its number of edges peaked during 2002-2007 to 16,000 from 1986-1988 which had 2,000 edges. (During the same time, the clustering coefficient has increased by 1/3.) The average distance has not decreased by much, which is surprising given the amount of increase in vertices.
  % To-do: figure out what clustering coefficient means
  \item UK - Number of averages vertices (675) is greater than the average number of edges (663). This means that in general, a node has slightly less than one connection. The average excess degree is greater than 200, which is greater than 100 times the average degree, which is roughly 2.0. This can also be explained when observing a subset of the betweenness graph of UK. The majority of the UK betweenness graph has a single node that is connected to many other nodes that only has one edge. 
  \image[0.2]{Images/uk-betweenness.png}{}{}{}
  \item Denmark - there are 2 themes that have the most number of vertices, which are economy and justice. According to Forbes in December 2017, Denmark is known for its extremely efficient economy and equitable distribution of income. It is also currently undergoing an economic expansion. Denmark is also rated \#1 for least corruption \cite{denmark}. Given this information, it is expected that Denmark would have the most legislators responsible for Economy and Justice.
  \item Switzerland - there exists few changes in Switzerland's graph statistics over the years. The country's clustering coefficient remains consistently around 0.5, and its number of vertices around 130 and its number of edges around 1750. 
  \item Sweden - there is an interesting change in the number of cosponsorships across parties between the years 2014-2018. The number of vertices across years consistently hover around 350, whereas the number of edges dropped from 4000 in 1988-1991 to 2000 in 2014-2018. This result happens to coincide with an event during the 2014 Swedish general election, in which the Social Democrats declared that they would not work with the Sweden Democrats. This decrease in edges across parties is evidently seen when comparing the assortativity graphs from the years 1988-1991 and 2014-2018.
  \image[0.3]{../plots/graph/sweden/assortativity/assortativity_net_se1988-1991.pdf}{}{}{}
  \image[0.3]{../plots/graph/sweden/assortativity/assortativity_net_se2014-2018.pdf}{}{}{}
  \item US - The US had extremely consistent number of edges (around 100,000), vertices (a little over 500), clustering coefficient (almost 1.0), average degree (around 350), and average excess degree (roughly 400). This shows that although members change every 2 years, the legislators and their interactions are relatively stable. 
\end{itemize}

\subsection{Across-Country Results}
\begin{itemize}
  \item Comparison of Number of Edges and Number of Vertices: UK has the largest number of vertices but one of the smallest number of edges (large number of people but very few connections made between them in comparison to other countries)
  \item Average Degree: US has average degree which is much higher than the average degree for other countries (each node is very highly connected to many other nodes)
  \item Comparison of Average Degree and Average Excess Degree: UK has one of the lowest average degrees but has the second largest average excess degree. This means that although each node is not connected to many other nodes, it is on average connected to other important nodes. This is a contrast between US and Denmark where their average degree and average excess degrees are relatively similar. 
  \item Comparison of Number of Edges and Average Distance: US has a much larger average number of edges when compared to other countries but has the second smallest average distance (there are many nodes but they are very highly connected so the distance between any two random nodes is much less than other countries)
  \item Comparison of Size of LCC and Number of CC: US has the largest average size of LCC but has the smallest number of CC (US nodes are very highly connected and only has a few connected components). In comparison, France has an average size of LCC that is slightly lower than the average of all countries, but it has the highest number of CC.
  \item US graph is almost completely connected. Fully connected graphs have $n(n-1)/2$ edges. There are about 100,000 edges and 552 vertices. A complete graph would have 152,076. So this is about 67\%.
\end{itemize}


\section{Discussion}
For our project, we use legislative cosponsorship networks across the US and selected European countries to try to infer social relationships in government that may influence legislative behavior. How well officials work together has a direct impact on the efficiency and productivity of government. The dynamic social relationships between legislators - of forging new relationships and maintaining older established relationships - matter.
We have uncovered some intriguing results. First, we notice that the average distance tends to increase as the world gets smaller. 

there exists an inverse relationship between the size of a graph and its average distance. 

\newpage
\begin{thebibliography}{9}

\bibitem{briatte} Briatte, Francois. ``Bill cosponsorship networks in European parliaments.'' 2016. \url{https://github.com/briatte/parlnet}.

\bibitem{denmark} ``Denmark.'' \emph{Forbes}, Forbes Magazine, \url{www.forbes.com/places/denmark/}

\bibitem{fowler1} Fowler, James H. ``Connecting the Congress: A Study of Cosponsorship Network.'' 2006. \url{http://fowler.ucsd.edu/best_connected_congressperson.pdf}.

\bibitem{fowler} Fowler, J. H., Waugh, A.S., and Sohn, Y. ``Cosponsorship Network.'' 2010. \url {http://jhfowler.ucsd.edu/cosponsorship.htm}

\bibitem{UK} UK Parliament. \url{https://edm.parliament.uk/}

\bibitem{US} United States Congress. \url{https://github.com/unitedstates/congress}
\end{thebibliography}



\end{document}
