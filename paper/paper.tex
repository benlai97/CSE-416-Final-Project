%
% Generic Template
% Created on 11/6/2017 by Jonathan Chen
%

\documentclass[12pt]{article}
\usepackage[left=0.8in,right=0.8in,top=0.8in,bottom=0.8in]{geometry}
\usepackage[utf8]{inputenc}
\usepackage{fancyhdr}                  % for header
\usepackage{graphicx}                  % for figures
\usepackage{amsmath}                   % for extented math markup
\usepackage{amssymb}
\usepackage{mathtools}
\usepackage{minted}
\usepackage{enumitem}
\usepackage{csquotes}
\usepackage{caption}
\usepackage{lipsum}
\usepackage[bookmarks=false]{hyperref} % for URL embedding
\usepackage{hyperref}

%%%%%%%%%%%%%%%%%%%%%%%%%%%%%%%%%%%%%%%%%%%%%%%%%%%%%%%%%%%%%%%%%%%%%%%%%%%%%%%%

\newcommand{\course}{CSE416A}
\newcommand{\names}{Benjamin Lai, Cate Jiang, and Jonathan Chen}

% create sheader and footer for every page
\pagestyle{fancy}
\fancyhf{}
\lhead{\textbf{Legislative Complex Networks}}
\chead{}
\rhead{Lai, Jiang, and Chen}
\cfoot{\thepage}

% environment for cases
\newlength\mylen
\settowidth\mylen{\textbf{Case~5.}}
\newlist{enumcases}{enumerate}{1}
\setlist[enumcases,1]{label=\textbf{Case~\arabic*.},leftmargin=0pt,align=right,%
    labelwidth=\dimexpr-\mylen-\labelsep\relax,}

% images
\captionsetup[figure]{
  font={footnotesize,singlespacing},
  labelfont=bf,
  singlelinecheck=off,
  justification=raggedright,
  labelsep=period
}

\newcommand{\image}[5][0.6]{
  % 1 - path to asset
  % 2 - label
  % 3 - title
  % 4 - caption
  \begin{figure}[H]
    \centering
    \includegraphics[width=1.0\textwidth,height=#1\textwidth,keepaspectratio]{#2}
    \caption[(details)]{\textbf{#4}. #5}
    \label{fig:#3}
  \end{figure}}


% minted
\usemintedstyle{pastie}
\setminted[python]{frame=lines,linenos,autogobble,framesep=1.4em,breaklines,%
                   escapeinside=||,mathescape,python3}

%%%%%%%%%%%%%%%%%%%%%%%%%%%%%%%%%%%%%%%%%%%%%%%%%%%%%%%%%%%%%%%%%%%%%%%%%%%%%%%%
% the document begins here!

\title{Comparative Analysis of Legislative Complex Networks of Select North American and European Countries}
\author{\names \\ \small\url{https://github.com/jowch/CSE-416-Final-Project}}
  
\date{December 11, 2018}
    
\begin{document}
\maketitle
% \vspace{0.1em}
\thispagestyle{empty}

\begin{abstract}
    Given the complex nature of today's political climate, we were interested in analyzing and quantifying various legislative bodies in select first-world countries. To do so, we performed complex network analysis on seven North Amerian and European countries (Austria, Denmark, France, Sweden, Switerland, UK, and US) to understand how a wide variety political parties interacted amongst each other with respect to cosponshorship. We found and plotted summary statistics and centrality measures to make comparisons possible.
\end{abstract}

\section{Introduction} 

In the United States, politics has increasingly become a point of conversation due to change of presidency and new bills being signed into laws. Cosponshorship provides a rich network between legislators that can be used to explain the political effects and efficiency of legislative processes. We propose to analyze the networks of different legislative bodies in select North American and European countries to determine how legislators interact amongst each other to cosponsor bills. We aim to answer questions such as which countries' legislators are most connected, how legislative parties and other attributes play into cosponsorship data, and what that means in terms of government performance and productivity. To perform our research, we will use cosponsorship data from different countries followed by analysis and comparison of statistics and centrality measures.

According a study conducted by James Folwer from the Department of Political Science at the University of California, Davis, there are several trends that have been observed among legislators. In the United States, biennial elections cause members of the U.S. House and Senate to change every two years, but the trends remain stable. Firstly, party affiliation and similar values and ideals often lead to increased possibility of cosponsorship. Second, average number of cosponsors per legislator reflects the degree that the individual is integrated into their political network \cite{fowler1}. Because these studies were performed solely within U.S. Congression, we propose to analyze legislators from different first-world countries to determine if these trends would be observed in other political environments as well.
 
\section{Materials and Methods}

\subsection{Data Collection and Pre-Processing}
We first obtained cosponsorship data for the countries: Austria, Denmark, France, Sweden, Switzerland, United Kingdom, and the United States. This was taken taken from the following resources: \underline{Bill cosponshorship} \cite{briatte}, \underline{Cosponsorship Network} \cite{fowler}, \underline{UK Parliament} \cite{UK}, and \underline{US Congress} \cite{US}. The raw data was parsed and saved as \texttt{graphml} files. Because the data from many databases were divided based on years, we also had to load and combine various \texttt{graphml} files to create a single network for each country we were analyzing. The data is stored in \texttt{CSE-416-Final-Project/data/} and the pre-processing was performed in the \texttt{Analysis} file in the root directory within our repository.

\subsection{Data Analysis}


\section{Results}
\begin{itemize}
  \item Comparison of Number of Edges and Number of Vertices: UK has the largest number of vertices but one of the smallest number of edges (large number of people but very few connections made between them in comparison to other countries)
  \item Average Degree: US has average degree which is much higher than the average degree for other countries (each node is very highly connected to many other nodes)
  \item Comparison of Average Degree and Average Excess Degree: UK has one of the lowest average degrees but has the second largest average excess degree. This means that although each node is not connected to many other nodes, it is on average connected to other important nodes. This is a contrast between US and Denmark where their average degree and average excess degrees are relatively similar.
  \item Comparison of Number of Edges and Average Distance: US has a much larger average number of edges when compared to other countries but has the second smallest average distance (there are many nodes but they are very highly connected so the distance between any two random nodes is much less than other countries)
  \item Comparison of Size of LCC and Number of CC: US has the largest average size of LCC but has the smallest number of CC (US are very highly connected and only has a few connected components). In comparison, France has an average size of LCC that is slightly lower than the average of all countries, but it has the highest number of CC.
\end{itemize}


\section{Discussion}

\newpage
\begin{thebibliography}{9}

\bibitem{briatte} Briatte, Francois. ``Bill cosponsorship networks in European parliaments.'' 2016. \url{https://github.com/briatte/parlnet}.

\bibitem{fowler1} Fowler, James H. ``Connecting the Congress: A Study of Cosponsorship Network.'' 2006. \url{http://fowler.ucsd.edu/best_connected_congressperson.pdf}

\bibitem{fowler} Fowler, J. H., Waugh, A.S., and Sohn, Y. ``Cosponsorship Network.'' 2010. \url {http://jhfowler.ucsd.edu/cosponsorship.htm}

\bibitem{UK} UK Parliament. \url{https://edm.parliament.uk/}

\bibitem{US} United States Congress. \url{https://github.com/unitedstates/congress}
\end{thebibliography}



\end{document}
